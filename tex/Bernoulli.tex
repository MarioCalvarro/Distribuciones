\section{Bernoulli}
\label{sec:bernoulli}
Distribución que mide la probabilidad de que un experimento acabe en ``éxito'',
con posibilidad $p$, o ``fracaso''. 
\[
\boxed{Ber\left( p \right)}
\]

\subsection{Función de masa}
La función de masa de la distribución es:
\[
p_X \left( x \right) = p^x \left( 1-p \right)^{1-x},\ x \in \left\{ 0, 1 \right\}
\]

\subsection{Función de distribución}
La función de distribución es:
\[
F_X\left( x \right) = \begin{cases}
    0, &x < 0\\ 
    1 - p, &0 \le x < 1\\
    1, &x \le 1 \\
\end{cases}
\]

\subsection{Momentos}

\subsubsection*{Respecto del origen}
La \textbf{esperanza} es: 
\[
    E\left[ X \right] = p
\]
y un momento genérico: 
\[
    E\left[ X^k \right] = p
\]
\subsubsection*{Respecto del centro}
La \textbf{varianza} es:
\[
    V\left[ X \right] = p\left( 1-p \right)
\]

\subsection{Función característica}
La función característica de la distribución es:
\[
\varphi\left( t \right) = \left( 1-p \right) + p \cdot \exp\left\{ it \right\}
\]

%\subsection{Otras características de interés}
