\section{Chi Cuadrado}
\label{sec:chi_cuadrado}
Sean $X_i \sim N\left( 0, 1 \right)$ con $i \in \left\{ 1, \ldots, n \right\}$ definimos entonces:
\[
\chi^2_n \sim \sum_{i=1}^{n} X_i
\]
como distribución \textbf{chi cuadrado de Pearson} con $n$ grados de libertad.

\subsection{Función de masa}
La función de densidad de la distribución es:
\[
f_X \left( x \right) = \frac{1}{2^{n/2}}\Gamma\left( \frac{n}{2} \right)\exp\left\{ -\frac{x}{2} \right\} x^{\frac{n}{2} - 1},\ x > 0
\]

\subsection{Función de distribución}
La función de distribución es:
\[
F_X\left( x \right) = \frac{1}{\Gamma\left( n / 2 \right)}\gamma\left( \frac{n}{2}, \frac{x}{2} \right)
\]

\subsection{Momentos}

\subsubsection*{Respecto del origen}
La \textbf{esperanza} es: 
\[
    E\left[ X \right] = n
\]
\subsubsection*{Respecto del centro}
La \textbf{varianza} es:
\[
    V\left[ X \right] = 2n
\]

\subsection{Función característica}
La función característica de la distribución es:
\[
\varphi\left( t \right) = \left( 1 - 2it \right)^{-n/2}
\]

\subsection{Otras características de interés}
\begin{itemize}
    \item Si tenemos $Y_1 \sim \chi^2_{n_1}$ y $Y_2 \sim \chi^2_{n_2}$. Entonces:
    \[
    Y_1 + Y_2 \sim \chi^2_{n_1 + n_2}
    \]
    \item Si tenemos $X \sim \chi^2_n$ y tomamos una $\mas\left( k \right)$. Entonces:
    \[
    \overline{X} \sim \gamma\left( nk, \frac{k}{2} \right)
    \]
\end{itemize}
