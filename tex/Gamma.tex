\section{Gamma}
\label{sec:gamma}
Distribución que mide la probabilidad de que en un tiempo $a$ ocurran $p$ eventos. (Puede que el tiempo sea $\frac{1}{a}$)
\[
\boxed{\gamma\left( p, a \right)}
\]

\subsection{Función de masa}
La función de densidad de la distribución es:
\[
f_X \left( x \right) = \frac{a^p}{\Gamma\left( p \right)} x^{p - 1} e^{-ax}
\]

\subsection{Función de distribución}
La función de distribución es:
\[
F_X\left( x \right) = \frac{1}{\Gamma\left( p \right)} \gamma\left( p, ax \right)
\]
(Poco importante)

\subsection{Momentos}

\subsubsection*{Respecto del origen}
La \textbf{esperanza} es: 
\[
    E\left[ X \right] = \frac{p}{a}
\]
\subsubsection*{Respecto del centro}
La \textbf{varianza} es:
\[
    V\left[ X \right] = \frac{p}{a^2}
\]

\subsection{Función característica}
La función característica de la distribución es:
\[
\varphi\left( t \right) = \left( 1 - \frac{it}{a} \right)^{-p}
\]

\subsection{Otras características de interés}
\begin{itemize}
    \item Si tenemos $X_i \sim \gamma\left( p_i, a \right)$ para $i \in \left\{ 1, \ldots, n \right\}$. Entonces:
    \[
    \sum_{i=1}^{n} X_i \sim \gamma\left( \sum_{i=1}^{n} p_i, a \right) 
    \]
    \item Si $X \sim \gamma\left( p, a \right) \Rightarrow$
    \[
    c X \sim \gamma\left( p, \frac{a}{c} \right),\ c \in \mathbb{R}
    \]
    \item $\gamma\left( 1, a \right) \equiv \exp\left( a \right)$
\end{itemize}
