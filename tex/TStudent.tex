\section{T-Student}
\label{sec:t_student}
Sean $Z \sim N\left( 0, 1 \right)$ e $Y \sim \chi^2_n$ definimos entonces:
\[
\boxed{t_{n-1} \sim \frac{Z}{\sqrt{\faktor{Y}{n}}}}
\]
como distribución \textbf{t-Student} con $n$ grados de libertad.

\subsection{Función de densidad}
La función de densidad de la distribución es:
\[
f_T \left( x \right) = \frac{1}{\sqrt{n \pi}}\frac{\Gamma\left( \frac{n + 1}{2} \right)}{\Gamma\left( \frac{n}{2} \right)} \left( 1 + \frac{x^2}{n} \right)^{-\frac{n+1}{2}}
\]

\subsection{Momentos}

\subsubsection*{Respecto del origen}
La \textbf{esperanza} es: 
\[
    E\left[ X \right] = 0
\]
\subsubsection*{Respecto del centro}
La \textbf{varianza} es:
\[
    V\left[ X \right] = \frac{n}{n-2}
\]

\subsection{Teorema de Student}
Sea $X \sim N\left( \mu, \sigma \right)$ y tomamos una $\mas\left( n \right)$ de la variable aleatoria. 
Entonces:
\[
\boxed{\sqrt{n}\frac{\overline{X} - \mu}{S} \sim t_{n-1}}
\]
o lo que es lo mismo: $\sqrt{n-1} \frac{\overline{X} - \mu}{\sqrt{b_2}} \sim t_{n-1}$.
%\subsection{Otras características de interés}
