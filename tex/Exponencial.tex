\section{Exponencial}
\label{sec:exponencial}
Distribución que mide la probabilidad que una cantidad $x$ de tiempo haya pasado entre 
dos eventos de una distribución Poisson $\lambda$.
\[
\boxed{\exp\left( a \right)}
\]

\subsection{Función de densidad}
La función de densidad de la distribución es:
\[
f_X \left( x \right) = \lambda e^{-\lambda x} \cdot I_{\left( 0, +\infty \right)}\left( x \right)
\]

\subsection{Función de distribución}
La función de distribución es:
\[
F_X\left( x \right) = 1 - e^{-\lambda x} \cdot I_{\left( 0, +\infty \right)}\left( x \right)
\]

\subsection{Momentos}

\subsubsection*{Respecto del origen}
La \textbf{esperanza} es: 
\[
    E\left[ X \right] = \lambda^{-1}
\]
y un momento genérico: 
\[
    E\left[ X^k \right] = \frac{k!}{\lambda^k}
\]
\subsubsection*{Respecto del centro}
La \textbf{varianza} es:
\[
    V\left[ X \right] = \lambda^{-2}
\]

\subsection{Función característica}
La función característica de la distribución es:
\[
\varphi\left( t \right) = \frac{\lambda}{\lambda - it}
\]

\subsection{Otras características de interés}
\begin{itemize}
    \item $\exp\left( a \right) \equiv \gamma\left( 1, a \right)$
    \item Si tenemos $X_i \sim \exp\left( a \right)$ para $i \in \left\{ 1, \ldots, n \right\}$. Entonces:
    \[
    \sum_{i=1}^{n} X_i \sim \exp \left( a \right) 
    \]
\end{itemize}
