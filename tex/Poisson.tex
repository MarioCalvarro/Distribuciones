\section{Poisson}
\label{sec:poisson}
Distribución que mide la probabilidad de que ocurran $x$ eventos, que tienen una ``velocidad'' $\lambda$, en un determinado intervalo de tiempo.
\[
\boxed{P\left( \lambda \right)}
\]

\subsection{Función de masa}
La función de masa de la distribución es:
\[
p_X \left( x \right) = \frac{\lambda^x \exp\left\{ -\lambda \right\}}{x!},\ x \in \mathbb{N}_0
\]

\subsection{Función de distribución}
La función de distribución es:
\[
F_X\left( x \right) = \exp\left\{ -\lambda \right\}\sum_{j=0}^{\left\lfloor x \right\rfloor} \frac{\lambda^j}{j!} 
\]
Poco importante.

\subsection{Momentos}

\subsubsection*{Respecto del origen}
La \textbf{esperanza} es: 
\[
    E\left[ X \right] = \lambda
\]
\subsubsection*{Respecto del centro}
La \textbf{varianza} es:
\[
    V\left[ X \right] = \lambda
\]

\subsection{Función característica}
La función característica de la distribución es:
\[
\varphi\left( t \right) = \exp\left\{ \lambda\left( e^{it} - 1 \right) \right\}
\]

\subsection{Otras características de interés}
\begin{itemize}
    \item Si tenemos $X_i \sim P\left( \lambda_i \right)$ para $i \in\left\{ 1, \ldots, n \right\}$. Entonces:
    \[
    \sum_{i=1}^{n} X_i \sim P\left( \sum_{i=1}^{n} \lambda \right) 
    \]
    %No del todo.
    \item Si tenemos una binomial, con número de ``éxitos'' esperados se mantiene más o menos constante, y hacemos tender $n$, número de casos, a infinito, tenemos como resultado una Poisson con $\lambda = np$.
\end{itemize}
